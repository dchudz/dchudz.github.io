\documentclass[letterpaper]{resume}
\usepackage{paralist}
\usepackage{hyperref}

\author{David J. Chudzicki}
\email{dchudz@gmail.com}
\citystatezip{Somerville, MA 02144}
\webpage{http://davidchudzicki.com}
\phone{(518) 366 7303}
\streetaddress{69 Morisson Avenue}
\github{https://github.com/dchudz}
\begin{document}

\maketitle

\section{Work Experience}

\affiliation[Data Scientist]
{Sense Labs}
{September 2015 - present}


\begin{compactitem}
\item Detecting individual electrical devices from house-wide electricity usage.
\end{compactitem}


\affiliation[Data Scientist \& Software Engineer]
{Kaggle (7\textsuperscript{th} employee)}
{January 2012 - August 2015}            


\begin{compactitem}
\item Introduced optimization-focused competitions at Kaggle (such as the Traveling Santa Problem), which have now become a staple of the site.
\item Designed Kaggle's infrastructure for automated feature creation and model fitting/evaluating across a range of models and feature sets (for our oil \& gas consulting arm of the business).
\item Spearheaded the use of interactive (Shiny) apps for exploratory visualization and visualizing the output of models. After some refinement, we used these apps in conversations with customers, leading customers to better understand and benefit from our work.
\end{compactitem}


\affiliation[Predictive Modeler]
{Allstate Insurance}
{October 2011 - January 2012}
{Associate Predictive Modeler}
{April 2010 - October 2011}


\begin{compactitem}
\item Built predictive models as the basis for Allstate's homeowners and auto insurance rating plans.
\item Initiated and managed development of a tool to help the measure the effect of implementing a rating plan with deviations from the model.
\end{compactitem}

\section{Education}

\affiliation[M.S. in Mathematics]
{University of Chicago}
{October 2007 - March 2010}
{}
{}
\begin{compactitem}
\item Probability and statistics, functional analysis, stochastic differential equations, applied topology.
\item Topic presentation: \textit{Dynamical Systems: Bifurcation Under Symmetry}.      
\end{compactitem}


\affiliation[B.A. in Mathematics \& Statistics, minor in Linguistics (GPA: 3.94)]
{Swarthmore College}
{September 2003 - June 2007}
{}
{}

\section{Projects}
\begin{compactitem}
\item \textbf{Predcomps (\href{http://www.davidchudzicki.com/predcomps/}{http://www.davidchudzicki.com/predcomps/})}: an R package for implementing Andrew Gelman's "average predictive comparisons", an approach for determining influence of input variables in a potentially complicated predictive model (e.g. without easily interpretable coefficients). Gelman mentioned the package \href{http://andrewgelman.com/2014/06/17/average-predictive-comparisons-r-david-chudzicki-writes-package/}{\underline{on his blog}}.
\end{compactitem}

\begin{compactitem}
\item \textbf{Simulated Knitting (\href{http://blog.davidchudzicki.com/2011/11/simulated-knitting.html}{http://blog.davidchudzicki.com/2011/11/simulated-knitting.html})}: Represented knitted objects in Python, embedded these objects in 3D, and visualized them.
\end{compactitem}

\section{Academic Research}

\affiliation[Graduate Research]
{Department of Mathematics, University of Chicago}
{April 2009 - March 2010}
{}
{}
\begin{compactitem}
\item Worked to apply differential geometry and 
statistics to problems in computer science. Investigated learning methods for 
high-dimensional (esp. semi-supervised) settings that exploit a lower-dimensional manifold structure by finding a good 
basis for functions which are smooth on the manifold.
\end{compactitem}

\affiliation[Research Assistant]
{Mathematics and Statistics Deptartment, Swarthmore College}
{June - August, 2007}
{}
{}
\begin{compactitem}
\item Assisted with the development and evaluation of estimators for the
date of species' extinction given an incomplete fossil record. We used
both analytical techniques and simulation.

\begin{compactitem}

\item Publication: S. Wang, D. Chudzicki, and P. Everson, 2009. ``Optimal Estimators of the Position of a Mass Extinction When Recovery Potential is Uniform.'' Paleobiology 35:3, pp. 447-459. 

\end{compactitem}

\end{compactitem}

\section{Talks}

\begin{compactitem}
\item \textbf{PyCon 2015, Montreal} (Tutorial): \textit{Winning Machine Learning Competitions With Scikit-Learn}
\item \textbf{Open Data Science Conference 2015, Boston}: \textit{\href{http://opendatascicon.com/schedule/data-workflows-for-iteration-collaboration-and-reproducibility/}{Data Workflows for Iteration, Collaboration, and Reproducibility}}
\begin{compactitem}
\item A talk about using GNU Make to organize a reproducible data science pipeline. \href{http://www.davidchudzicki.com/slides/odsc-2015-workflow/}{(slides)}.
\end{compactitem}
\end{compactitem}




\section{Programming}
\vspace{1pt}
\begin{compactitem}
\item \textbf{Languages} 
\begin{compactitem}
\item Extensive experience: Python, R.
\item Some experience: C\#, JavaScript, Julia, MATLAB/Octave, SQL.
\end{compactitem}
\item \textbf{Data visualization}: D3, ggplot2.
\end{compactitem}

\end{document}
