\documentclass[letterpaper]{resume}
\usepackage{paralist}
\usepackage{hyperref}

\author{David J. Chudzicki}
\email{dchudz@gmail.com}
\citystatezip{Somerville, MA 02144}
\webpage{http://davidchudzicki.com}
\phone{(518) 366 7303}
\streetaddress{69 Morisson Avenue}
\github{https://github.com/dchudz}
\begin{document}

\maketitle

\section{Work Experience}


\affiliation[Software Engineer]
{Wave}
{December 2016 - June 2017}


\begin{compactitem}
\item Fought fraud with heuristics and machine learning; later, led engineering for the international transfers team.
\end{compactitem}



\affiliation[Software Engineer]
{DataRobot}
{December 2015 - December 2016}


\begin{compactitem}
\item Expanded DataRobot's customer-facing API for interacting with its machine learning product by designing and implementing backend API routes, corresponding (Python and R) client package functionality, and providing API guidance to other teams implementing new functionality. I also helped customers see the value in automating their workflows, leading the API to shift from an afterthought to a major priority for the company.
\end{compactitem}


\affiliation[Data Scientist]
{Sense Labs}
{September 2015 - December 2015}


\begin{compactitem}
\item Detecting individual electrical devices from house-wide electricity usage.
\end{compactitem}


\affiliation[Data Scientist \& Software Engineer]
{Kaggle (7\textsuperscript{th} employee)}
{January 2012 - August 2015}            


\begin{compactitem}
\item Designed Kaggle's infrastructure for feature creation and fitting/evaluating models across a range of model types, feature sets, geography, and validation method (for our oil \& gas consulting arm of the business).
\item Spearheaded the use of interactive (Shiny) apps for exploratory visualization and visualizing the output of models. After some refinement, we used these apps in conversations with customers, leading customers to better understand and benefit from our work.
\end{compactitem}


\affiliation[Predictive Modeler]
{Allstate Insurance}
{October 2011 - January 2012}
{Associate Predictive Modeler}
{April 2010 - October 2011}


\begin{compactitem}
\item Built predictive models as the basis for Allstate's homeowners and auto insurance rating plans.
\item Proposed/prototyped a tool to measure the effect of implementing a rating plan with deviations from the model.
\end{compactitem}

\section{Education}

\affiliation[M.S. in Mathematics]
{University of Chicago}
{October 2007 - March 2010}
{}
{}
\begin{compactitem}
\item Probability and statistics, functional analysis, stochastic differential equations, applied topology.
\item Topic presentation: \textit{Dynamical Systems: Bifurcation Under Symmetry}.      
\end{compactitem}


\affiliation[B.A. in Mathematics \& Statistics, minor in Linguistics (GPA: 3.94)]
{Swarthmore College}
{September 2003 - June 2007}
{}
{}

\section{Academic Research}

\affiliation[Graduate Research]
{Department of Mathematics, University of Chicago}
{April 2009 - March 2010}
{}
{}
\begin{compactitem}
\item Investigated machine learning methods for 
high-dimensional (esp. semi-supervised) settings that exploit a lower-dimensional manifold structure by finding a good 
basis for functions which are smooth on the manifold.
\end{compactitem}

\affiliation[Research Assistant]
{Mathematics and Statistics Deptartment, Swarthmore College}
{June - August, 2007}
{}
{}
\begin{compactitem}

\item Publication: S. Wang, D. Chudzicki, and P. Everson, 2009. ``Optimal Estimators of the Position of a Mass Extinction When Recovery Potential is Uniform.'' Paleobiology 35:3, pp. 447-459. 

\end{compactitem}

\section{Talks}

\begin{compactitem}
\item \textbf{Open Data Science Conference 2017, Santa Clara}: data science tutorials (R, Python)
\item \textbf{PyCon 2015, Montreal} (Tutorial): \textit{Winning Machine Learning Competitions With Scikit-Learn}
\item \textbf{Open Data Science Conference 2015, Boston}: \textit{\href{http://opendatascicon.com/schedule/data-workflows-for-iteration-collaboration-and-reproducibility/}{Data Workflows for Iteration, Collaboration, and Reproducibility}}
\begin{compactitem}
\item A talk about using GNU Make to organize a reproducible data science pipeline. \href{http://www.davidchudzicki.com/slides/odsc-2015-workflow/}{(slides)}.
\end{compactitem}
\end{compactitem}


\section{Programming}
\vspace{1pt}
\begin{compactitem}
\item Extensive experience: Python (especially Python 3!), R.
\item Some experience: Java, C\#, JavaScript, Julia, MATLAB/Octave, SQL.
\end{compactitem}

\end{document}
