	\documentclass[letterpaper]{resume}
\usepackage{paralist}
\usepackage{hyperref}

\author{David J. Chudzicki}
\email{dchudz@gmail.com}
\citystatezip{(518) 366 7303}
\webpage{http://davidchudzicki.com}
\phone{69 Morisson Avenue}
\streetaddress{Somerville, MA 02144}

\begin{document}

\maketitle

\section{Work Experience}

\affiliation[Data Scientist]
{Kaggle}
{January 2012 - present}            


\begin{compactitem}
\item \textbf{Competitions}: Planned and executed predictive modeling competitions to meet customers' goals. Identified the appropriate data, train/test split, and evaluation metric. Transformed data into the appropriate form and coded evaluation metrics for our production system.
\item \textbf{Oil \& Gas}: Assisted oil companies in optimizing well locations and drilling parameters. Set up the team's infrastructure for feature extraction, model evaluation. Implemented methods for visualizing the output of our machine learning models to extract insights.
\item \textbf{Kaggle Scripts}: Helped build \href{https://www.kaggle.com/scripts}{https://www.kaggle.com/scripts}, where Kaggle users can share and execute scripts without downloading data sets.
\end{compactitem}


\affiliation[Predictive Modeler]
{Allstate Insurance}
{October 2011 - January 2012}
{Associate Predictive Modeler}
{April 2010 - October 2011}


\begin{compactitem}
\item Built loss-cost predictive models as the basis for
Allstate's homeowners and auto insurance rating
plans. Helped train new hires in generalized linear models
and model validation. 
\item Initiated and managed
development of a tool to help the Pricing Department measure
the effect of implementing a rating plan with deviations from the model.
\end{compactitem}

\section{Education}

\affiliation[M.S. in Mathematics]
{University of Chicago}
{October 2007 - March 2010}
{}
{}
\begin{compactitem}
\item probability and statistics, functional analysis, stochastic differential equations, applied topology.
\item Topic presentation: \textit{Dynamical Systems: Bifurcation Under Symmetry}.      
\end{compactitem}


\affiliation[B.A. in Mathematics \& Statistics, minor in Linguistics]
{Swarthmore College}
{September 2003 - June 2007}
{}
{}
\begin{compactitem}
\item GPA: 3.94 
\end{compactitem}


\section{Academic Research Experience}

\affiliation[Graduate Research]
{Department of Mathematics, University of Chicago}
{April 2009 - March 2010}
{}
{}
\begin{compactitem}
\item Worked to apply differential geometry and 
statistics to problems in computer science. Investigated learning methods for 
high-dimensional (esp. semi-supervised) settings that exploit a lower-dimensional manifold structure by finding a good 
basis for functions which are smooth on the manifold.
\end{compactitem}

\affiliation[Research Assistant]
{Mathematics and Statistics Deptartment, Swarthmore College}
{June - August, 2007}
{}
{}
\begin{compactitem}
\item Assisted with the development and evaluation of estimators for the
date of species' extinction given an incomplete fossil record. We used
both analytical techniques and simulation.

\begin{compactitem}

\item Publication: S. Wang, D. Chudzicki, and P. Everson, 2009. ``Optimal Estimators of the Position of a Mass Extinction When Recovery Potential is Uniform.'' Paleobiology 35:3, pp. 447-459. 

\end{compactitem}

\end{compactitem}

\section{Talks}

\begin{compactitem}
\item \textbf{PyCon 2015, Montreal} (Tutorial): \textit{Winning Machine Learning Competitions With Scikit-Learn}
\item \textbf{Open Data Science Conference 2015, Boston}: \textit{\href{http://opendatascicon.com/schedule/data-workflows-for-iteration-collaboration-and-reproducibility/}{Data Workflows for Iteration, Collaboration, and Reproducibility}}
\begin{compactitem}
\item A talk about using GNU Make to organize a reproducible data science pipeline. \href{http://www.davidchudzicki.com/slides/odsc-2015-workflow/}{(slides)}.
\end{compactitem}
\end{compactitem}




\section{Programming}
\vspace{1pt}
\begin{compactitem}
\item \textbf{Languages} 
\begin{compactitem}
\item Extensive: Python, R
\item Moderate: C\#, JavaScript, Julia, MATLAB/Octave, SQL
\end{compactitem}
\item \textbf{Data visualization}
\begin{compactitem}
\item D3, ggplot2.
\end{compactitem}
\end{compactitem}

\end{document}
